%提出するレポートの書式はこのtemplateファイルに沿って作成してください.
%特に表紙・概要の書式は変えないで下さい.

\documentclass[a4j]{jarticle}
\usepackage{listings,jlisting}
%\usepackage{jarticle}
%\usepackage[dvips]{graphicx}
\usepackage{fancybox}
\usepackage[dvipdfmx]{graphicx}
\usepackage{epsbox}
\usepackage{url}
\usepackage{here}
\usepackage[hang,small,bf]{caption}
\usepackage[subrefformat=parens]{subcaption}
\captionsetup{compatibility=false}

\setlength{\headsep}{-5mm}
\setlength{\oddsidemargin}{0mm}
\setlength{\textwidth}{165mm}
\setlength{\textheight}{230mm}
\setlength{\footskip}{20mm}

\title{
\vspace{30mm}
{\bf 情報学群実験第4i レポート}
\\
\vspace{5mm}
第5回\\
\vspace{5mm}
{\bf 画像処理における画像成分の扱い}
\vspace{90mm}
}

\author{
\vspace{5mm}
グループ6 \\
\vspace{5mm}
学籍番号 1190361 \\
\vspace{5mm}
{\large 早川 晋矢}
\vspace{10mm}
}

\begin{document}

\maketitle

\newpage



\section{p-タイル法}
\label{sect:ouyou10_2}
\begin{verbatim}
  %p-タイル法
  %読み込む画像の切り取る領域を58%と仮定

  %-----初期化処理-----
  clear;
  %-----画像読み出し-----
  img = imread('usagi01.png');         %画像の読み込み
  [y,x,z] = size(img);                 %画像のサイズ(y=縦座標,x=横座標,z=RGB)
  %-----RGB値の取り出し-----
  r = double(img(:,:,1));              %R値取得
  g = double(img(:,:,2));              %G値取得
  b = double(img(:,:,3));              %B値取得
  %-----グレースケール化-----
  gray = 0.3*r+0.59*g+0.11*b;          %グレイスケール化
  %-----降順ソート-----
  sdata = sort(gray(:),'descend');     %降順にソート
  %-----閾値設定-----
  p = 0.58                             %p値を58%と指定する
  t = sdata(round((x*y)*p));           %39150番目の画素値を閾値にする
  %-----2値化処理-----
  two_color = zeros(y,x);              %ゼロ配列を作成
  two_color(gray>=t) = 255;            %閾値以下の画素値に255(白)を格納
  %-----画像表示-----
  figure(4);
  imshow(two_color);
\end{verbatim}

\section{モード法}
\begin{verbatim}
  %モード法

  %-----初期化処理-----
  clear;
  %-----画像読み出し-----
  img = imread('tutiusa.jpg');         %画像の読み込み
  [y,x,z] = size(img);                 %画像のサイズ(y=縦座標,x=横座標,z=RGB)
  %-----RGB値の取り出し-----
  r = double(img(:,:,1));              %R値取得
  g = double(img(:,:,2));              %G値取得
  b = double(img(:,:,3));              %B値取得
  %-----グレースケール化-----
  gray = 0.3*r+0.59*g+0.11*b;          %グレイスケール化
  imwrite(uint8(gray),'guretuti.png');
  %-----閾値設定-----
  t = 140;
  %-----2値化処理-----
  two_color = zeros(y,x);              %ゼロ配列を作成
  two_color(gray<=t) = 255;            %閾値以下の画素値に255(白)を格納
  %-----画像表示-----
  figure(4);
  imshow(two_color);
  imwrite(two_color,'mode.png');
\end{verbatim}

\section{判別分析法}
\begin{verbatim}
  %判別分析法

  %-----初期化処理-----
  clear;
  %-----画像読み出し-----
  img = imread('tutiusa.jpg');           %画像の読み込み
  [y,x,z] = size(img);                   %画像のサイズ(y=縦座標,x=横座標,z=RGB)
  %-----RGB値の取り出し-----
  r = double(img(:,:,1));                %R値取得
  g = double(img(:,:,2));                %G値取得
  b = double(img(:,:,3));                %B値取得
  %-----グレースケール化-----
  gray = 0.3*r+0.59*g+0.11*b;            %グレイスケール化
  %-----ヒストグラム化------
  gray8 = uint8(gray);                   %画素値を0から255の整数に
  for k = 0:255                          %画素値を0から255まで探索
      data = (gray8 == k);               %探索値と同じ画素値を探索
      data = sum(sum(data));             %見つけた画素値の個数を数える
      histData(k+1) = data;              %配列に格納する
  end
  %-----閾値設定-----
  max_t = 0;                             %クラス間分散が最大となる閾値の保存用
  max_val = 0;                           %最大となるクラス間分散の保存用
  for t = 0:255                          %閾値を0から255まで探索
      %---初期値設定---
      w1 = 0;                            %黒側クラスの画素数
      w2 = 0;                            %白側クラスの画素数
      sum1 = 0;                          %黒側クラス合計値
      sum2 = 0;                          %白側クラス合計値
      m1 = 0;                            %黒側クラスの平均
      m2 = 0;                            %白側クラスの平均
      %---黒側クラスの画素数,画素値---
      for k = 0:t                        %閾値までが黒側クラス
          w1 = w1 + histData(t+1);       %画素数を足し合わせる
          sum1 = sum1 + k*histData(t+1); %画素値を足し合わせる
      end
      %---白側クラスの画素数,画素値---
      for k = t:255                      %閾値以降が白側クラス
          w2 = w2 + histData(t+1);       %画素数を足し合わせる
          sum2 = sum2 + k*histData(t+1); %画素値を足し合わせる
      end
      %---ゼロ除算を防ぐ---
      if (w1==0 | w2==0)
          continue;                      %次の閾値判定へ
      end
      %---クラス別の平均値の算出---
      m1 = sum1/w1;                      %黒側クラスの平均算出
      m2 = sum2/w2;                      %白側クラスの平均算出
      %---クラス間の分散を算出---
      result = w1*w2*(m1-m2)*(m1-m2) / ((w1+w2)*(w1+w2)); %式(9.2)
      %---クラス間分散値の更新---
      if (max_val < result)
          max_val = result;
          max_t = t;
      end
  end
  %-----2値化処理-----
  two_color = zeros(y,x);                %ゼロ配列を作成
  two_color(gray>=max_t) = 255;          %閾値以下の画素値に255(白)を格納
  %-----画像表示-----
  figure(5);
  imshow(two_color);
\end{verbatim}

\section{輪郭追跡}
\begin{verbatim}
  %配列格納順番
  %searchX,searchY,entryCodeはこの順番に基づいている
  %[1,2,3
  % 8,-,4
  % 7,6,5]
  %進入方向番号
  %searchX,searchY,entryCodeはこの順番に基づいている
  %[0,1,2
  % 7,-,3
  % 6,5,4]

  %-----初期化処理-----
  clear;
  %-----画像読み出し-----
  im = imread('two_mohu.png');         %画像の読み込み
  [y,x,z] = size(im);                 %画像のサイズ(y=縦座標,x=横座標,z=RGB)
  %-----画像周辺に画素追加-----
  img = uint8(ones(y+2,x+2).*255);    %画像より一回り大きい白(255)配列作成
  img(2:y+1,2:x+1) = im;              %読み込んだ画像に適用
  %-----初期値設定-----
  searchValue = 0;                    %探索で探すべき値の設定
  searchX = [-1,0,1,1,1,0,-1,-1];     %3*3探索の際の横座標調整用
  searchY = [-1,-1,-1,0,1,1,1,0];     %3*3探索の際の縦座標調整用
  entryCode = [5,6,7,0,1,2,3,4];      %進入方向から座標算出用
  tracked = zeros(y+2,x+2);           %追跡済み座標格納用
  entryDirection = zeros(y+2,x+2);    %進入方向保存用(更新は1本の輪郭追跡ごとの初回)
  check = zeros(y+2,x+2);             %追跡済み座標格納用(1本の輪郭追跡ごとにリセット)
  boxSize = 8;                        %探索を行う範囲のサイズ
  modVal = boxSize;                   %余りを求める際に使用
  loopbox = boxSize-1;                %画素周辺の探索の繰り返し回数
  exit = 0;                           %処理の終了用スイッチ
  %-----目標画素値の探索-----
  for m=2:y-1
      for n=2:x-1
      %-----目標画素値であり,左画素が反する画素であり,未追跡画素を見つけるまでループ-----
      if (img(m,n) ~= searchValue | img(m,n-1) == searchValue | tracked(m,n) ~= 0);
          continue;
      end
      %-----輪郭追跡で称する変数の初期化-----
      nextEntry = 1;                  %現在の画素にどの方向から進入したか
      entryDirection(m,n) = nextEntry;%右から進入したことを格納
      tracked(m,n) = 1;               %探索済みと更新
      sx = n;                         %x座標のコピー
      sy = m;                         %y座標のコピー
      exit = 0;                       %処理の終了用スイッチ
      check = zeros(y+2,x+2);         %追跡済み座標格納用(1本の輪郭追跡ごとにリセット)
      %-----輪郭追跡-----
      while (exit==0)                 %終了条件を満たすまで続ける
          for k=0:loopbox             %画素周辺の探索
              %-----探索座標の算出-----
              boxX = sx+searchX(mod(k+nextEntry,modVal)+1);
              boxY = sy+searchY(mod(k+nextEntry,modVal)+1);
              %-----連続画素の探索-----
              if img(boxY,boxX) == searchValue;
                  %-----現在の輪郭追跡で探索済みか判定-----
                  if check(boxY,boxX) == 0;
                      %-----未探索なら進入方向、探索判定、次の進入方向の更新-----
                      tracked(boxY,boxX) = 1;
                      check(boxY,boxX) = 1;
                      entryDirection(boxY,boxX) = entryCode(mod(k+nextEntry,modVal)+1);
                      nextEntry = entryCode(mod(k+nextEntry,modVal)+1);
                  elseif entryDirection(boxY,boxX) == entryCode(mod(k+nextEntry,modVal)+1)
                      %-----一周できたことを確認したら終了-----
                      exit=1;
                  else
                      %-----次の進入方向のみ更新-----
                      nextEntry = entryCode(mod(k+nextEntry,modVal)+1);
                  end
                  %-----座標を更新して次の探索へ-----
                  sx = boxX;
                  sy = boxY;
                  break;
              end
              %-----1ドットが検出された際はここで終了する-----
              if(k ==loopbox)
                  exit=1;
              end
          end
      end
  end
  end
  %-----画像の出力-----
  figure(5);
  result=ones(y+2,x+2);
  result=(result-tracked)*255;
  resultImg=result(2:y-1,2:x-1);
  imshow(resultImg);
  imwrite(resultImg,'track2.png');
\end{verbatim}

\section{収縮・膨張処理}
\begin{verbatim}
  %-----初期化処理-----
  clear;
  %-----画像読み出し-----
  im = imread('inp.png');         %画像の読み込み
  [y,x,z] = size(im);             %画像のサイズ(y=縦座標,x=横座標,z=RGB)
  %-----オープニング処理-----
  im = erosion(im);               %収縮処理
  im = dilation(im);              %膨張処理
  %-----クロージング処理-----
  im = dilation(im);              %膨張処理
  im = erosion(im);               %収縮処理
  %-----画像表示-----
  figure(5);
  imshow(im);
  imwrite(im,'opclo.png');

%----------収縮用関数erosion---------
  function resultImg = erosion(im)
  %-----画像読み出し-----
  %im = imread(name);                 %画像の読み込み
  [y,x,z] = size(im);                 %画像のサイズ(y=縦座標,x=横座標,z=RGB)
  %-----画像周辺に画素追加-----
  img = uint8(ones(y+2,x+2).*255);    %画像より一回り大きい白(255)配列作成
  img(2:y+1,2:x+1) = im;              %読み込んだ画像に適用
  %result = uint8(ones(y+2,x+2).*255);%結果を格納する配列を作成する
  result = uint8(zeros(y+2,x+2));
  boxSize = 8;                        %探索を行う範囲のサイズ
  %-----初期値設定-----
  searchValue = 0;                    %探索で探すべき値の設定
  checkValue = 255;                   %周囲で確認する値の設定
  searchX = [-1,0,1,1,1,0,-1,-1];     %3*3探索の際の横座標調整用
  searchY = [-1,-1,-1,0,1,1,1,0];     %3*3探索の際の縦座標調整用
  %-----目標画素値の探索-----
  for m=2:y-1
      for n=2:x-1
          %-----探索画素が見つかるまでループ-----
          if (img(m,n) ~= searchValue)
              continue;
          end
          %-----周辺画素探索用変数の初期化-----
          sx = n;                              %x座標のコピー
          sy = m;                              %y座標のコピー
          for k=1:boxSize                      %画素周辺の探索
              boxX = n+searchX(k);             %座標の算出
              boxY = m+searchY(k);             %座標の算出
              if img(boxY,boxX) == checkValue; %探索する値が見つかれば
                  result(m,n) = checkValue;    %更新する値を代入
                  break;
              end
          end
      end
  end
  %-----画像の出力-----
  %figure(5);
  result=uint8(result+img);
  resultImg=result(2:y-1,2:x-1);
  %imshow(resultImg);

%----------膨張用関数dilation---------

  %function resultImg = dilation(name)
  function resultImg = dilation(im)
  %-----画像読み出し-----
  %im = imread(name);                 %画像の読み込み
  [y,x,z] = size(im);                 %画像のサイズ(y=縦座標,x=横座標,z=RGB)
  %-----画像周辺に画素追加-----
  img = uint8(ones(y+2,x+2).*255);    %画像より一回り大きい白(255)配列作成
  img(2:y+1,2:x+1) = im;              %読み込んだ画像に適用
  result = uint8(ones(y+2,x+2));      %結果を格納する配列を作成する
  boxSize = 8;                        %探索を行う範囲のサイズ
  %-----初期値設定-----
  searchValue = 255;                  %探索で探すべき値の設定
  checkValue = 0;                     %周囲で確認する値の設定
  searchX = [-1,0,1,1,1,0,-1,-1];     %3*3探索の際の横座標調整用
  searchY = [-1,-1,-1,0,1,1,1,0];     %3*3探索の際の縦座標調整用
  %-----目標画素値の探索-----
  for m=2:y-1
      for n=2:x-1
          %-----探索画素が見つかるまでループ-----
          if (img(m,n) ~= searchValue)
              continue;
          end
          %-----周辺画素探索用変数の初期化-----
          sx = n;                              %x座標のコピー
          sy = m;                              %y座標のコピー
          for k=1:boxSize                      %画素周辺の探索
              boxX = n+searchX(k);             %座標の算出
              boxY = m+searchY(k);             %座標の算出
              if img(boxY,boxX) == checkValue; %探索する値が見つかれば
                  result(m,n) = checkValue;    %更新する値を代入
                  break;
              end
          end
      end
  end
  %-----画像の出力-----
  %figure(5);
  result=uint8(result.*img);
  resultImg=result(2:y-1,2:x-1);
  %imshow(resultImg);
\end{verbatim}

\section{ラベリング}
\begin{verbatim}
  %-----初期化処理-----
  clear;
  %-----画像読み出し-----
  im = imread('two_mohu.png');             %画像の読み込み
  [y,x,z] = size(im);                     %画像のサイズ(y=縦座標,x=横座標,z=RGB)
  %-----画像周辺に画素追加-----
  img = uint8(ones(y+2,x+2).*255);        %画像より一回り大きい白(255)配列作成
  img(2:y+1,2:x+1) = im;                  %読み込んだ画像に適用
  %-----初期値設定-----
  searchValue = 0;                        %探索で探すべき値の設定
  white = 255;
  searchX = [-1,0,1,
             -1,0,1,
             -1,0,1];
  searchY = [-1,-1,-1,
              0, 0, 0,
              1, 1, 1];
  lutSize = x*y;
  label = zeros(y+2,x+2);                 %追跡済み座標格納用
  lookupTable = [];         %同一連結成分の設定
  %lookupTable = updataLUT(lookupTable,7,12)
  lutpoint = 1;
  nextlabel = 1;
  %-----目標画素値の探索-----
  c = 'check';
  for m=2:y+1
      for n=2:x+1
          %-----探索画素が見つかるまでループ-----
          if img(m,n) ~= searchValue
              continue;
          end
          %-----ラベルの更新-----
          for k = [1,2]%左上と上のラベル確認
              if label(m+searchY(k),n+searchX(k)) ~= 0
                  label(m,n) = label(m+searchY(k),n+searchX(k));
                  break;
              end
          end
          if label(m,n) ~= 0%注目ラベルがゼロでなければ
              for k = [2,3,4]%上と右上と左
                  if label(m+searchY(k),n+searchX(k)) ~= 0 & label(m+searchY(k),n+searchX(k)) ~= label(m,n)
                  %-----lookpuTable更新-----
                  lookupTable = updataLUT(lookupTable,label(m,n),label(m+searchY(k),n+searchX(k)));
                  end
              end
              if label(m+searchY(3),n+searchX(3)) ~= 0 & img(m+searchY(6),n+searchX(6)) == white & label(m+searchY(k),n+searchX(k)) ~= label(m,n)
                  %-----lookpuTable更新-----
                  lookupTable = updataLUT(lookupTable,label(m,n),label(m+searchY(3),n+searchX(3)));
              end
              pretrue = 1;
              for k = [1,2]%左上と上
                  if img(m+searchY(k),n+searchX(k)) ~= white
                      pretrue = 0;
                      break;
                  end
              end
              if pretrue == 1 & label(m+searchY(4),n+searchX(4)) ~= 0
              %lookpuTable更新----------------------------------
              lookupTable = updataLUT(lookupTable,label(m,n),label(m+searchY(4),n+searchX(4)));
              end
          end
          pretrue = 1;
          for k = [1,2]%左上と上と左
              if img(m+searchY(k),n+searchX(k)) ~= white
                  pretrue = 0;
                  break;
              end
          end
          if pretrue == 1
              label(m,n) = nextlabel;
              nextlabel = 1+nextlabel;
          end
      end
  end
  %-----ルックアップテーブル更新作業-----
  renVal = zeros(nextlabel-1,nextlabel-1);
  for k = 1 : nextlabel-1
      rvc = 1;
      search = find(lookupTable == k);
      for l = search'; %'
          if l <= length(lookupTable)
              l = l + length(lookupTable);
          else
              l = l - length(lookupTable);
          end
          save = 1;
          for m = 1:rvc-1
              if renVal(m) == 0
                  break;
              end
              if renVal(k,m) == lookupTable(l)
                  save = 0;
                  break;
              end
          end
          if save == 1
              renVal(k,rvc) = lookupTable(l);
              rvc = rvc + 1;
          end
      end
  end

  lookupTable2 = zeros(nextlabel-1,nextlabel-1);
  valSet = [1:nextlabel-1];
  exit = 1;
  while true
      for k = 1 : nextlabel-1
              min = valSet(k);
          for m = 1 : nextlabel-1
              if renVal(k,m) == 0
                  break;
              end
              if renVal(k,m) < min
                  min = renVal(k,m);
                  exit = 0;
              end
              if valSet(renVal(k,m)) < min
                  min = valSet(renVal(k,m));
                  exit = 0;
              end
          end
          if min ~= valSet(k)
              valSet(k) = min;
          end
      end
      if exit == 1
          break
      end
      exit = 1;
  end

  type = unique(valSet);
  for m = 1:length(type)
      for n = 1:length(valSet)
          if valSet(n) == type(m)
              valSet(n) = m;
          end
      end
  end

  %
  for m=2:y+1
      for n=2:x+1
          if label(m,n) ~= 0
              label(m,n) = valSet(label(m,n));
          end
      end
  end

  maxLabel = max(valSet);

  result = zeros(y,x,3);
  for m=2:y+1
      for n=2:x+1
          if label(m,n) ~= 0
              result(m,n,1) = label(m,n)/maxLabel;
              result(m,n,2) = 1;
              result(m,n,3) = 1;
          else

              result(m,n,3) = 1;
          end
      end
  end

  resultImg = hsv2rgb(result);
  imshow(resultImg);
  imwrite(resultImg,'labelinged.png');

  function lookupTable = updataLUT(lookupTable,valA,valB)
  [ty, tx] = size(lookupTable);

  if ty == 0
      lookupTable = [lookupTable;valA, valB];
  else
      for t = 1:ty
          if lookupTable(t,1) == valA
              if lookupTable(t,2) == valB
                  break;
              end
          elseif lookupTable(t,2) == valA
              if lookupTable(t,1) == valB
                  break;
              end
          end
          if t == ty
              lookupTable = [lookupTable;valA, valB];
          end
      end
  end
\end{verbatim}

\section{細線化処理}
\begin{verbatim}
  %[P9][P2][P3]
  %[P8][P1][P4]
  %[P7][P6][P5]
  %

  %-----初期化処理-----
  clear;
  %-----画像読み出し-----
  im = imread('two_mohu.png');             %画像の読み込み
  [y,x,z] = size(im);                     %画像のサイズ(y=縦座標,x=横座標,z=RGB)
  %-----画像周辺に画素追加-----
  img = uint8(ones(y+2,x+2).*255);        %画像より一回り大きい白(255)配列作成
  img(2:y+1,2:x+1) = im;                  %読み込んだ画像に適用
  searchValue = 0;
  white = 255;
  black = 0;
  searchX = [0,0,1,1,1,0,-1,-1,-1];
  searchY = [0,-1,-1,0,1,1,1,-1,-1];
  searchData = img;
  changeData = img;
  changed = 0;
  while true
      %-----ステップ1-----
      for m=2:y+1
          for n=2:x+1
              %-----探索画素が見つかるまでループ-----
              if searchData(m,n) ~= black
                  continue;
              end
              if f1(searchData,m,n) ~= 1
                  continue;
              end
              if f2(searchData,m,n) < 2 | f2(searchData,m,n) > 6
                  continue;
              end
              if searchData(m+searchY(2),n+searchX(2))==black & searchData(m+searchY(4),n+searchX(4))==black & searchData(m+searchY(6),n+searchX(6))==black
                  continue;
              end
              if searchData(m+searchY(4),n+searchX(4))==black & searchData(m+searchY(6),n+searchX(6))==black & searchData(m+searchY(8),n+searchX(8))==black
                  continue;
              end
              changeData(m,n) = white;
              changed = 1;
          end
      end
      searchData = changeData;

      %-----ステップ2-----
      for m=2:y+1
          for n=2:x+1
              %-----探索画素が見つかるまでループ-----
              if searchData(m,n) ~= black
                  continue;
              end
              if f1(searchData,m,n) ~= 1
                  continue;
              end
              if f2(searchData,m,n) < 2 | f2(searchData,m,n) > 6
                  continue;
              end
              if searchData(m+searchY(2),n+searchX(2))==black & searchData(m+searchY(4),n+searchX(4))==black & searchData(m+searchY(8),n+searchX(8))==black
                  continue;
              end
              if searchData(m+searchY(2),n+searchX(2))==black & searchData(m+searchY(6),n+searchX(6))==black & searchData(m+searchY(8),n+searchX(8))==black
                  continue;
              end
              changeData(m,n) = white;
              changed = 1;
          end
      end
      searchData = changeData;
      if changed == 0;
          break;
      end
      changed = 0;
  end

  imshow(searchData);
  imwrite(searchData,'thinninged.png')

  function result = f1(searchData,m,n)
  searchX = [0,0,1,1,1,0,-1,-1,-1];
  searchY = [0,-1,-1,0,1,1,1,-1,-1];

%-----f1関数-----
  white = 255;
  black = 0;
  count = 0;
  beforeVal = black;
  for k = [2,3,4,5,6,7,8,9,2]
      if searchData(m+searchY(k),n+searchX(k)) == black
          if beforeVal == white
              count = count + 1;
          end
          beforeVal = black;
      else
          beforeVal = white;
      end
  end
  result = count;

%-----f2関数-----
  function result = f2(searchData,m,n)
  searchX = [0,0,1,1,1,0,-1,-1,-1];
  searchY = [0,-1,-1,0,1,1,1,-1,-1];

  white = 255;
  black = 0;
  count = 0;
  beforeVal = black;
  for k = [2,3,4,5,6,7,8,9]
      if searchData(m+searchY(k),n+searchX(k)) == black
          count = count + 1;
      end
  end
  result = count;



\end{verbatim}


\end{document}
